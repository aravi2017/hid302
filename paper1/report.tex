\documentclass[sigconf]{acmart}

\usepackage{hyperref}

\usepackage{endfloat}
\renewcommand{\efloatseparator}{\mbox{}} % no new page between figures

\usepackage{booktabs} % For formal tables

\settopmatter{printacmref=false} % Removes citation information below abstract
\renewcommand\footnotetextcopyrightpermission[1]{} % removes footnote with conference information in first column
\pagestyle{plain} % removes running headers

\begin{document}
\title{Big Data Application in Restaurant Industry}


\author{Sushant Athaley}
\affiliation{%
  \institution{Indiana University}
}
\email{sathaley@iu.edu}



\begin{abstract}
This paper provides insight into how big data can be used in restaurant industry. It will also explore various challenges using big data in the restaurant industry. 
\end{abstract}

\keywords{big data, restaurant, application, analytics}


\maketitle

\section{Introduction}

The \textit{proceedings} are the records of a
conference. ACM seeks to give these
conference by-products a uniform, high-quality appearance.  To do
this, ACM has some rigid requirements for the format of the
proceedings documents: there is a specified format (balanced double
columns), a specified set of fonts (Arial or Helvetica and Times
Roman) in certain specified sizes, a specified live area, centered on
the page, specified size of margins, specified column width and gutter
size.

\section{The Body of The Paper}

Typically, the body of a paper is organized into a hierarchical
structure, with numbered or unnumbered headings for sections,
subsections, sub-subsections, and even smaller sections.  The command
\texttt{{\char'134}section} that precedes this paragraph is part of
such a hierarchy. \LaTeX\ handles the
numbering and placement of these headings for you, when you use the
appropriate heading commands around the titles of the headings.  If
you want a sub-subsection or smaller part to be unnumbered in your
output, simply append an asterisk to the command name.  Examples of
both numbered and unnumbered headings will appear throughout the
balance of this sample document.

Because the entire article is contained in the \textbf{document}
environment, you can indicate the start of a new paragraph with a
blank line in your input file; that is why this sentence forms a
separate paragraph.

\subsection{Type Changes and {\itshape Special} Characters}

We have already seen several typeface changes in this sample.  You can
indicate italicized words or phrases in your text with the command
\texttt{{\char'134}textit}; emboldening with the command
\texttt{{\char'134}textbf} and typewriter-style (for instance, for
computer code) with \texttt{{\char'134}texttt}.  But remember, you do
not have to indicate typestyle changes when such changes are part of
the \textit{structural} elements of your article; for instance, the
heading of this subsection will be in a sans serif\footnote{Another
  footnote here.  Let's make this a rather long one to see how it
  looks. Footnotes must be avoided.} typeface, but that is handled by
the document class file.  Take care with the use of the curly braces
in typeface changes; they mark the beginning and end of the text that
is to be in the different typeface.

You can use whatever symbols, accented characters, or non-English
characters you need anywhere in your document; you can find a complete
list of what is available in the \textit{\LaTeX\ User's Guide}
\cite{Lamport:LaTeX}.

\subsection{Math Equations}

You may want to display math equations in three distinct styles:
inline, numbered or non-numbered display.  Each of
the three are discussed in the next sections.

\subsubsection{Inline (In-text) Equations}

A formula that appears in the running text is called an
inline or in-text formula.  It is produced by the
\textbf{math} environment, which can be
invoked with the usual \texttt{{\char'134}begin\,\ldots{\char'134}end}
construction or with the short form \texttt{\$\,\ldots\$}. You
can use any of the symbols and structures,
from $\alpha$ to $\omega$, available in
\LaTeX~\cite{Lamport:LaTeX}; this section will simply show a
few examples of in-text equations in context. Notice how
this equation:

\begin{math}
  \lim_{n\rightarrow \infty}x=0
\end{math},

set here in in-line math style, looks slightly different when
set in display style.  (See next section).


\section{Conclusions}

This paragraph will end the body of this sample document.  Remember
that you might still have Acknowledgments or Appendices; brief samples
of these follow.  There is still the Bibliography to deal with; and we
will make a disclaimer about that here: with the exception of the
reference to the \LaTeX\ book, the citations in this paper are to
articles which have nothing to do with the present subject and are
used as examples only.



\appendix

%Appendix A
\section{Headings in Appendices}

The rules about hierarchical headings discussed above for the body of
the article are different in the appendices.  In the \textbf{appendix}
environment, the command \textbf{section} is used to indicate the
start of each Appendix, with alphabetic order designation (i.e., the
first is A, the second B, etc.) and a title (if you include one).  So,
if you need hierarchical structure \textit{within} an Appendix, start
with \textbf{subsection} as the highest level. Here is an outline of
the body of this document in Appendix-appropriate form:

\subsection{Introduction}
\subsection{The Body of the Paper}
\subsubsection{Type Changes and  Special Characters}
\subsubsection{Math Equations}
\paragraph{Inline (In-text) Equations}
\paragraph{Display Equations}
\subsubsection{Citations}
\subsubsection{Tables}
\subsubsection{Figures}
\subsubsection{Theorem-like Constructs}
\subsubsection*{A Caveat for the \TeX\ Expert}
\subsection{Conclusions}
\subsection{References}

Generated by bibtex from your \texttt{.bib} file.  Run latex, then
bibtex, then latex twice (to resolve references) to create the
\texttt{.bbl} file.  Insert that \texttt{.bbl} file into the
\texttt{.tex} source file and comment out the command
\texttt{{\char'134}thebibliography}.

% This next section command marks the start of
% Appendix B, and does not continue the present hierarchy

\section{More Help for the Hardy}

Of course, reading the source code is always useful.  The file
\path{acmart.pdf} contains both the user guide and the commented code.

\begin{acks}

  The authors would like to thank Dr. Yuhua Li for providing the
  matlab code of the \textit{BEPS} method.

  The authors would also like to thank the anonymous referees for
  their valuable comments and helpful suggestions. The work is
  supported by the \grantsponsor{GS501100001809}{National Natural
    Science Foundation of
    China}{http://dx.doi.org/10.13039/501100001809} under Grant
  No.:~\grantnum{GS501100001809}{61273304}
  and~\grantnum[http://www.nnsf.cn/youngscientsts]{GS501100001809}{Young
    Scientsts' Support Program}.

\end{acks}

\bibliographystyle{ACM-Reference-Format}
\bibliography{report} 

\end{document}
