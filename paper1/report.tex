\documentclass[sigconf]{acmart}

\usepackage{hyperref}

\usepackage{endfloat}
\renewcommand{\efloatseparator}{\mbox{}} % no new page between figures

\usepackage{booktabs} % For formal tables

\settopmatter{printacmref=false} % Removes citation information below abstract
\renewcommand\footnotetextcopyrightpermission[1]{} % removes footnote with conference information in first column
\pagestyle{plain} % removes running headers

\begin{document}
\title{Big Data Application in Restaurant Industry}


\author{Sushant Athaley}
\affiliation{%
  \institution{Indiana University}
}
\email{sathaley@iu.edu}



\begin{abstract}
Paper provides insight into how big data can be used in the restaurant industry. It also explores how big data can be collected and analyzed so that it helps restaurant industry to do better in profit margins and give their customer a great hospitality experience. Paper will try to find out current technologies and solutions available in big data processing for the restaurant industry. It will also focus on various challenges involved in using big data in the restaurant business. It is a review/research paper which considers information from various sources like articles, books and web to provide the information. 
\end{abstract}

\keywords{i523, hid302, big data, restaurant, application, analytics}


\maketitle

\section{Introduction}
Big data is revolutionizing the way business is getting conducted in various industries. The retailer like Amazon uses it to provide personalized buying suggestions and social networking site like LinkedIn uses it to connect more people. Question is, do we have big data available for the restaurant industry and how big data application is going to be beneficial. The restaurant industry is facing challenges like shrinking labor pool, moderate economic growth, costly labor, challenging profit margin, high competition, moderate sales growth and growing expectation from the customer on the dining experience, can big data application help overcome these challenges.\cite{www-restaurant-challenges}

The paper is structured as follows. Section "Ingredient" captures various data points available in the restaurant industry for the big data analysis. Section "Consume" provides details on how data can be gathered in the restaurant industry. Section "Recipe for Success" captures various big data analysis which can help to solve different problems. Section "Solution and Tools Available" provides information on current big data solutions and tools available for the restaurant industry. Section "Flavourful Implementation" provides real-life examples of big data applications in the restaurant industry. Section "Hell's Kitchen" capture various challenges involved in using big data for the restaurant industry. Finally, section "Conclusion" concludes the paper.


\section{Big Data for Restaurant/Ingredient}
To understand how big data analytics will help, we first need to find out what are the data points present in the restaurant industry which can be considered as big data. As one of the V-variety of big data, the restaurant also has structured and unstructured data. Structured data is something which is getting generated inside the restaurant and unstructured data is something which is outside of the restaurant. Refer figure 1.
\begin{figure}
\includegraphics{images/datasource}
\caption{Image courtesy restaurant org - Data Sources}
\end{figure}

\subsection{Structured Data}
Structured data is well formatted, easy to understand and analyze. Restaurant POS(point of sale) system shows what’s selling, where, and at what time\cite{www-qsr}. Food and beverage cost, labor cost, product mix, rent cost are obvious data points. Raw material required for preparation, menu, ingredient consideration, meal preparation, product availability from the supplier, prices of products comes from the kitchen of the restaurant. Staffing schedule, table turnover, bar management, wages, salaries, tips, customer feedback is data. The number of time employee coming late, number of times drinks provided as comp due to server error is data.\cite{www-restaurant}
\subsection{Unstructured Data}
Unstructured data is un-formatted, difficult to gather and analyze. Data shared from social media like trends, retweets, shares, and comments categorize as unstructured data. Customer promotions, customer profile like age, gender, address, email, taste preference, favorite dish, various milestones like birthdate, anniversary etc, along with family information is also an unstructured data. Weather and traffic information also constitutes as an important data to consider. \cite{www-restaurant}


\section{Collect Big Data/Consume}
These various data attributes can be collected from the different system. Most of the data is generated inside the restaurant by the system like POS which captures all sales transactions. POS system can also break down sales by time, size of the party, menu items, and ingredients. The inventory provides information on suppliers, food, beverages, and gas and electricity bill. Payroll provides information on wages, salaries, employee schedule, and time off by the employees. Loyalty program and marketing promotions provides data regarding marketing of the restaurant.


Outside data can be gathered through the various applications like OpenTable, Facebook, Twitter, Yelp, TripAdvisor, Foursquare, Urbanspoon or Instagram, weather and traffic sites. Information can be gathered from customer like his favorite menu/drink item, favorite table, special request, allergies, liking to the presentation, feedback on ambiance, service and food. \cite{www-restaurant}

\section{Big Data Analytics/Recipe for Success}
Benjamin Stanley, co-founder of Food Genius, suggests "A restaurant operator shouldn't just jump into big data unless they have a problem they are trying to solve"\cite{KooserAmandaC.2013BD}. Big data analytics can help with various analysis which can solve different issues but it's important to know the problem which needs to be solved. Menu analysis can help with deciding the cost of the item, popular menu item, how often items are ordered, the time when menu item ordered, ingredient used and if any ingredient needs to be substituted\cite{KooserAmandaC.2013BD}. Labor cost can be managed better by analyzing overtime pay, absenteeism, costs to sales, costs by department and server, tips, amount of time spent at the table, types of entrées sold and whether the server sells the special. This analysis can be used to motivate, train and provide incentives to the servers\cite{www-restaurant},\cite{KooserAmandaC.2013BD}. Guest check analytics can help determine what sells well, how often somebody orders certain items and detailed pricing analyses\cite{www-restaurant}. Customer profile analysis gives insight on demographics of the customer, ages, income level, their family information, kind of food they like, allergies, drink habits, places they dine out, special occasions and this analysis can be used to provide the personalized experience to the customer\cite{www-restaurant}. Servers can use customer profile analysis to suggest menu choices, celebrate birthdays or special occasions, or run specials to drive more business. Reservation system data analysis helps in understanding who all are coming, when they last visited, what they tend to order, are they celebrating any special occasion and accordingly then can decide on the menu\cite{www-bostonglobe}. Data mining of data from social media like Facebook, Twitter, Instagram, YouTube can help in understanding sentiments of the customer, social news, training topic, views on self and competitor restaurants, identify brand or restaurant fans\cite{JENNINGSLISA2015Mbds}. This mining also provides the capability to get feedback real time and respond at the same time. This information can be used to do targeted marketing for the specific audience\cite{JENNINGSLISA2015Mbds}. 

\section{Solution and Tools Available}
Fishbowl provides cost-effective data analytics solution to the restaurant industry using Hadoop and other technologies. Fishbowl integrated Hadoop with their marketing platform to provide guest analytics, menu management, media analytics, promotions and mobile platform to provide complete solutions.\cite{www-foodnewsfeed}\cite{www-fishbowl}  

MyCheck and MarketingVitals.com together provide mobility and data analytics platform for the hospitality industry. \cite{www-buss} 

Dickeys Barbecue Pit restaurant has worked with big data and business intelligence service providers iOLAP to develop a proprietary system it calls Smoke Stack. \cite{www-forbes}

Upserve, a restaurant management platform, provides payment processing, point of sale, data insights to boost margins and exceed guest expectations.\cite{www-bostonglobe}\cite{www-upserve}


\section{Real Life Examples/Flavorful Implementation}
A quickservice chain monitors its drive-thru lanes to determine which items to display on its digital menu board. When lines are longer, the menu features items that can be prepared quickly. When lines are shorter, the menu features higher-margin items that take a bit longer to prepare. Those subtle changes in the menu board wouldn't be possible if the company couldn't tap into a steady stream of data in real time to make instantaneous adjustments.\cite{www-restaurant}


Haute Dogs and Fries, a two-unit, quickservice restaurant in Alexandria, Va., leverages social media to connect with customers. Being small and community-focused allows the operation to quickly identify market trends and make offers in real-time, says co-owner Lionel Holmes. He monitors social media throughout the day and might post a lunch special at 11 a.m. or a dinner offer at 3 p.m. based on what is trending. Haute Dogs and Fries is on Twitter, Facebook and Instagram and uses email to reach customers and build loyalty.\cite{www-restaurant}


Fig and Olive, a seven-location New York-based restaurant group, has used guest-management software to track more than 500,000 guests and \$17.5 million in checks. The restaurants have been able to customize the dining experience for individual guests and deliver results with targeted email communications. It's ``we miss you campaig`` offered complimentary crostini to guests who hadn't dined there in 30 days. The result: Almost 300 visits and more than \$36,000 in sales, translating into a return of more than seven times the cost of the program. Matthew Joseph, who leads technology and information systems for the company, says linking POS data with online reservations, plus monitoring social media mentions on Facebook, Twitter or TripAdvisor, helped Fig and Olive create its brand identity and build loyalty.\cite{www-restaurant}


 Dickeys Barbecue Pit, which operates 514 restaurants across the U.S., uses Smoke Stack system to provide near real-time feedback on sales and other key performance indicators. All of the data is examined every 20 minutes to enable immediate decisions if the sale is not at certain baseline at a certain store in the region then it enables them to deploy training or operation directly to that store. For example, if there is lower than expected sales one lunchtime, and have an amount of ribs there, then text invitation is sent to people in the local area for ribs special – to both equalize the inventory and catch up on sales.\cite{www-forbes}

\section{Challenges of Using Big Data/Hell's Kitchen}
The restaurant industry is very slow in terms of adopting or spending on new technologies due to small profit margins, high employee turnover and the overall cost of implementation\cite{www-bostonglobe}. Most of the restaurants are still using legacy software packages which are inadequate in dealing with the big data. These legacy software packages are cumbersome to upgrade or integrate with new technologies or data streams which are required for the big data analytics. It can take a lot of times to get data from old restaurant software to the data warehouse. Even if data is centralized, it's difficult for most of the restaurants to hire a data scientist to analyze data due to their costly salaries. Only big restaurant chain can afford such costly labor and tools needed for the big data application\cite{2015BDLC}. Another major challenge is the variety of big data source and format involved in restaurant industry like structured data in form of POS, inventory systems and unstructured data like social networking site or weather reports. Combining data from such various sources is big deal. There are financial challenges also as technology offered to work with big data is expensive which makes leveraging big data challenging for most of the restaurants.\cite{www-foodnewsfeed}

\section{Conclusions}

Big data application offers ample opportunities to solve the various problems faced by the restaurant industry. It is opening avenues which cannot be imagined earlier but adoption of big data application is a bit slow in restaurant industry compared to other industries like retail due to low-profit margins and high application cost. Currently, big data is mostly used by the large chain and Michelin start restaurants who can afford the big data solutions. Efforts are getting made to provide low-cost solutions so that small and medium restaurant can also embrace the big data. There is no doubt that big data application is going to change the way people dine out and as quickly restaurant adopts it, it's going to provide customers that Umami effect.



\begin{acks}
  The authors would like to thank Dr. Gregor von Laszewski for his
  support and suggestions to write this paper.
\end{acks}


\bibliographystyle{ACM-Reference-Format}
\bibliography{report} 

\end{document}
